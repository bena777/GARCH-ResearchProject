\documentclass{article}
\usepackage{graphicx} % Required for inserting images
\usepackage{url}

\title{GARCH Models and Applications}
\author{Benjamin Adelman}
\date{December 2024}

\begin{document}

\maketitle

\section{Abstract}
This paper will investigate both the theoretical underlying of what a GARCH model actually is/trying to measure as well as potential real applications of the family of models.This paper will focus on determining which version of the standard GARCH model works the best in predicting the future volatility of various different asset classes. This will involve first understanding the components of the GARCH model (formulation, properties etc.), researching its potential applications in predicting volatility clustering in specific markets, then implementing it into code with data to test it on. Finally, statistical testing will be used to compare the different GARCH models between both each other and other volatility models (Exp weight moving average).
\section{Definitions}
An ARCH model uses past returns along with an error term to model future volatility. It is described in the form ARCH(p) where p is the number of lags and can be written as $$\sigma^2_t = \omega + \alpha_1\epsilon^2_{t-1} ... + \alpha_q \epsilon^2_{t-q}$$
A GARCH model is known as the "generalized ARCH model" and accounts for past variance in addition to past returns (like ARCH) and subsequent error terms to estimate future volatility. It can be written in the form  of GARCH(p,q) which is expressed with $$\sigma^2_t = \omega + \alpha_1 \epsilon^2_{t-1} +\beta_1 \sigma^2_{t-1} + … + \alpha_q \epsilon_{t-q}^2 + \beta_p \sigma^2_{t-p}$$
where $\omega$ is a random error term, $\epsilon_t$ is previous return at time t, $\sigma_t$ is the previous variance at time t, and $\omega, \alpha_i$ and $\beta_i$ are coefficients $>0$ \\
\\
 These coefficients are commonly found using the statistical concept of maximum likelihood estimate (MLE), which is the value of an arbitrary parameter $\theta$ that maximizes the probability of getting the observed data (Penn State, n.d.). It can be defined as $$L(\theta) = P(X_1=x_1,...X_n=x_n) = f(x_1;\theta) \cdot...f(x_n;\theta) = \Pi_{i=1}^nf(x_i;\theta)$$
 \\
 where $f(x_i; \theta)$ is the probability density function of each $X_i$.
 \\
 What both of these models are essentially trying to do is predict a future volatility by using past returns (or in the case of GARCH, past returns and past volatility). The GARCH model will be the main focus of this paper due to its unique ability to capture long term volatility patterns as well as model clustering, a phenomenon observed in financial markets (Bollerslev, 1986).\\
\section{Setup}
The coding portion of this research project will be explained on a high level, however the Github repo with the actual code used for testing will be linked at the end of the paper. The Python programming language was used to impliment this project along with the libraries "yfinance" to access equity data, "arch" to fit the models, as well as "matplotlib" and "statsmodels" to display results.  \\
\\
I chose to analyze three different assets for this project, each in a different asset class. The S\&P 500 to represent equities, WTI Crude Oil to represent commodities, and the EUR/USD currency pair for foreign exchange markets. I chose these three assets due to them being some of the most liquid in their respective asset classes. In particular, I chose oil due to its unique phenomenon such as extreme volatility and negative prices. The time frame used in this analysis will be from 1/01/2006 to 1/01/2024. The period of 1/01/2006-12/31/2019 will serve as the training partition while 1/01/2020-11/01/2024 will be the testing.
\section{Process}
The base program of this project starts by fitting a GARCH(1,1) model to the asset it is modeling with all the data from the training phase. Thankfully, Python's "arch" module takes care of the dirty work of parameter estimation using the aforementioned MLE. After the model is properly fit, its parameters are plugged into the "garch\_generalized" function for whatever model is currently being tested. It predicts the volatility for the duration of the testing phase. The predicted results are then compared to the actual results during the testing phase and summed up in the form of a mean squared error statistic. This process is then repeated for different GARCH models of various lengths. We also are able to describe the predicted volatility model's 95\% confidence interval by multiplying each predicted value by the appropriate z score $(z=1.96)$.
\begin{figure}[h!]
\centering
\includegraphics[width=\textwidth]{Figure_1.png}
\caption{A visual representation of the predicted volatility and confidence interval of a GARCH(1,1) model}
\end{figure}
\section{Results}
The first thing that was abruptly clear was that the order of magnitude of the GARCH model does not really change the accuracy of the prediction all that much. The mean squared error of different models (figure 2) for the same underlying asset were extremely similar with some minor fluctuations. Because of this, in the majority of cases it is practical to simply use the GARCH(1,1) model. This further confirms the academic consensus of it being the most widely used variation of the classic GARCH model (Jafari et al., 2007). Models can all be looked at from the trade off of complexity vs accuracy, meaning that in this case, the small improvement in accuracy is not worth the exponential increase in complexity.
\begin{figure}[h!]
\centering
\begin{tabular}{||c|c|c|c||}
\hline
Model & SPY & WTI & EUR/USD \\ [0.5ex]
\hline\hline
GARCH(1,1) & 3.3495 & 43.5812 & 0.4351 \\
\hline
GARCH(1,2) & 3.3493 & 43.4785 & 0.4345\\
\hline
GARCH(2,1) & & & \\  % Empty row, if you don't have data for this row
\hline
GARCH(2,2) & 3.3525 & 43.5926 & 0.4345 \\
\hline
GARCH(3,3) & 3.3531 & 43.4687 & 0.4346 \\
\end{tabular}
\caption{Model volatility estimates for SPY, WTI, and EUR/USD using different GARCH models.}
\end{figure}
\\
Another interesting result that was observed was the differing levels of accuracy of the model depending on which asset it was being used on.


\section{Future Steps}
This research is only the tip of the iceberg when it comes to GARCH models. There is literally hundreds of different models in the GARCH family each with their own unique properties. In the future, I would particularly like to explore the applications of the EGARCH model which takes into account asymmetric returns in its volatility predictions.
\section{References}
\begin{itemize}
\noindent
Engle, R. F. (2001). GARCH 101: An introduction to the use of ARCH/GARCH models in applied econometrics. \textit{Journal of Economic Perspectives}, \textbf{15}(4), 157–168.\\
\\
\noindent
Gondalia, V. (2018). Measuring conditional volatility using GARCH (2, 2) model from an empirical standpoint. \textit{NLDIMSR Innovision Journal of Management Research}, \textbf{2}(2), 35–38.\\
\\
\noindent
PennState: Statistics Online Courses. (n.d.). Maximum Likelihood Estimation | STAT 415. Retrieved from \url{https://online.stat.psu.edu/stat415/lesson/1/1.2}.\\
\\
\noindent
Bollerslev, T. (1986). Generalized autoregressive conditional heteroskedasticity. \textit{Journal of Econometrics}, \textbf{31}(3), 307–327. North-Holland.\\
\noindent
\\
Jafari, G. R., Bahraminasab, A., \& Norouzzadeh, P. (2007). Why does the standard GARCH(1, 1) model work well? \textit{International Journal of Modern Physics C}, 18(7), 1223–1230. \url{https://doi.org/10.1142/s0129183107011261}
\end{itemize}
\\

\end{document}
